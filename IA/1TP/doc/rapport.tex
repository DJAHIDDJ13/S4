\documentclass {article}

\usepackage[utf8]{inputenc}
\usepackage{graphicx}
\usepackage{amsmath}
\usepackage{mathtools}
\usepackage{float}

\title{TP1 IA}
\author{ABDELMOUMENE Djahid}

\begin{document}
\maketitle

\section{Fonction d'evaluation}
La fonction d'evaluation depend de deux critéres. Le premier c'est la
difference des sommes des valeurs des pions de joueur et l'adversaire.
Cette heuristique capture le fait qu'on a plus de chance de gagner si on
a plus des pions avec des valeur grandes.

On prend $N$ comme la nombre des lignes, $M$ nombre des colonnes et $Pions$
l'ensemble des pions, où pour chaque pion on peux récupérer la couleur ($1$ ou
$-1$), la valeur et les indices dans le plateau x et y.

\begin{equation*}
   valDiff(Pions) = \sum_{p \in Pions} p.col * p.val
\end{equation*}

Et le deuxiéme c'est la difference des sommes des distances inversés
(\textit{ie}: $N - dist$) vers la ligne de fond. La valeur de la distance est
inversé parce qu'on veux que l'evaluation soit grande quand les pions sont proche
de fond et petite lorsque les pions sont loin. Cette heuristique encourage les
pions a se rapprocher vers les pions de l'adversaire et vers la ligne de fond ou
on peux gagner.

\begin{equation*}
   indiceFond(couleur) =
   \begin{cases}
      N-1 & couleur = -1 = \text{o}\\
      0   & couleur = 1 = \text{x}
   \end{cases}
\end{equation*}

\begin{equation*}
   distDiff(Pions) = \sum_{p \in Pions} p.col * (N - \lvert p.x
   - indiceFond(p.col) \rvert)
\end{equation*}

Pour combiner ces deux critéres on choisit un facteur $\lambda$ pour multiplier
$valDiff$ et on fait la somme, cette valeur doit indiquer le facteur d'importance
de la $valDiff$ de la $distDiff$. C'est à dire qu'on veux prioriser l'attack
des pion de l'avancement si $\lambda > 1$.

Alors la fonction d'evaluation:

\begin{equation*}
   H(Pions, joueur) = joueur * (valDiff(Pions) * \lambda + distDiff(Pions))
\end{equation*}

\section{Complexité}
On calcule la fonction de coût $c(p)$ où $p$ est la profondeur maximale et
$F$ est le facteur de branchement - \textit{ie} moyenne des nombre des bouges
possible à chaque coup -, et $cst$ un constant décrivant les conditions et
operations unitaires pour effectuer le minimax.

\begin{equation*}
   c(p) =
   \begin{cases}
      N * M            &  p = 0 \\
      F * c(p-1) + cst &  p \geq 1
   \end{cases}
\end{equation*}

Si on prend $u(p) = c(p) - \frac{cst}{1-F}$ alors:
\begin{align*}
   u(p) &=
   \begin{cases}
      N*M - \frac{1}{1-F}              & p = 0 \\
      F * (c(p-1) - \frac{1}{1-F}) + cst & p \geq 1
   \end{cases}
\\ 
    &=
   \begin{cases}
      N*M - \frac{cst}{1-F}  & p = 0 \\
      F * c(p-1)
   \end{cases}
\end{align*}

Alors $U_p$ est un suite géometrique ou le terme génerale est:
\begin{align*}
   u(p) &= u(0) * F^p
        &= (N * M - \frac{cst}{1-F}) * F^p
\end{align*}

Alors on peux déduire $c(p)$:
\begin{align*}
   c(p) &= u(p) + \frac{cst}{1-F}\\
        &= (N * M - \frac{cst}{1-F}) * F^p + \frac{cst}{1-F}
\end{align*}

Alors la complexité est:
\begin{equation*}
   c(p) = \mathcal{O}(F^p)
\end{equation*}

Alors si on prend $F = 30$ - \textit{estimation empirique} -
\begin{equation}
   c(p) \approx 30^p
\end{equation}

\newpage
\section{Analyse expérimentale}
\begin{figure}[H]
  \includegraphics[width=\linewidth]{comp_nodes.pdf}
  \caption{Comparaison entre negamax et negamax avec l’élagage $\alpha/\beta$ au
   niveau de nombre des noeuds traités}
  \label{fig:comp_nodes}
\end{figure}

En théorie on doit avoir un gain moyenne de l'ordre de racine carrée pour la version
avec l'élagage alpha-bêta:
\begin{align*}
   c(p) &= \mathcal{O}(\sqrt{F^p}) \\
        &= \mathcal{O}(F^{\frac{p}{2}})
\end{align*}

\begin{figure}[H]
  \includegraphics[width=\linewidth]{comp_times.pdf}
  \caption{Comparaison entre negamax et negamax avec l’élagage $\alpha/\beta$ au
   niveau de temps d'éxecution}
  \label{fig:comp_times}
\end{figure}

Pour maximiser le nombre de coupure avec l'algorithme $\alpha/\beta$. On peut
utiliser une technique de la programmation dynamique qui s'appelle
la \textbf{Mémoïsation} ou on garde un tableau de mappage entre les noeuds et le
résultat de l'appel minimax. et a chaque fois qu'on essaye de calculer le
minimax d'un plateau, on cherche d'abord dans ce tableau si la noeud est
déja stocké on récupére directement le résultat. Mais cette technique ne
marchera pas toujours car elle consomme trop de memoire $\mathcal{O}(2^n)$
pour les jeux ayant un facteur de branchement grand.

Une autre optimisation c'est de trier les bouges possibles avant commencer, 
Ce tri dois prioritiser les bouges ayant plus de chance d'avoir un evaluation
minimax grande, pour cela on peux utiliser l'heuristique existant pour tries
ces bouges. Ca doit améliorer le performance puisque les bouges qui ont un
bonne heuristique dés les premier coups devront au moyenne avoir un heuristique
meilleur en generale, alors le nombre des coupures doit s'augmenter car
l'interval $[\alpha, \beta]$ sera rognés plus rapidement.


\end{document}
