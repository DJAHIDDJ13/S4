\documentclass {article}

\usepackage[utf8]{inputenc}
\usepackage{graphicx}
\usepackage{amsmath}
\usepackage{mathtools}

\title{TP1 IA}
\author{ABDELMOUMENE Djahid}

\begin{document}
\maketitle

\section{Fonction d'evaluation}
La fonction d'evaluation depend de deux critéres. Le premier c'est la
difference des sommes des valeurs des pions de joueur et l'adversaire.
Cette heuristique capture le fait qu'on a plus de chance de gagner si on
a plus des pions avec des valeur grandes.

On prend $N$ comme la nombre des lignes, $M$ nombre des colonnes et $Pions$
l'ensemble des pions, où pour chaque pion on peux récupérer la couleur ($1$ ou
$-1$), la valeur et les indices dans le plateau x et y.

\begin{equation*}
   valDiff(Pions) = \sum_{p \in Pions} p.col * p.val
\end{equation*}

Et le deuxiéme c'est la difference des sommes des distances inversés
(\textit{ie}: $N - dist$) vers la ligne de fond. La valeur de la distance est
inversé parce qu'on veux que l'evaluation soit grande quand les pions sont proche
de fond et petite lorsque les pions sont loin. Cette heuristique encourage les
pions a se rapprocher vers les pions de l'adversaire et vers la ligne de fond ou
on peux gagner.

\begin{equation*}
   indiceFond(couleur) =
   \begin{cases}
      N-1 & couleur = -1 = \text{o}\\
      0   & couleur = 1 = \text{x}
   \end{cases}
\end{equation*}

\begin{equation*}
   distDiff(Pions) = \sum_{p \in Pions} p.col * (N - \lvert p.x
   - indiceFond(p.col) \rvert)
\end{equation*}

Pour combiner ces deux critéres on choisit un facteur $\lambda$ pour multiplier
$valDiff$ et on fait la somme, cette valeur doit indiquer le facteur d'importance
de la $valDiff$ de la $distDiff$. C'est à dire qu'on veux prioriser l'attack
des pion de l'avancement si $\lambda > 1$.

Alors la fonction d'evaluation:

\begin{equation*}
   H(Pions, joueur) = joueur * (valDiff(Pions) * \lambda + distDiff(Pions))
\end{equation*}

\section{Complexité}
On calcule la fonction de coût $c(p)$ où $p$ est la profondeur maximale et
$F$ est le facteur de branchement - \textit{ie} moyenne des nombre des bouges
possible à chaque coup -, et $cst$ un constant décrivant les conditions et
operations unitaires pour effectuer le minimax.

\begin{equation*}
   c(p) =
   \begin{cases}
      N * M            &  p = 0 \\
      F * c(p-1) + cst &  p \geq 1
   \end{cases}
\end{equation*}

Si on prend $u(p) = c(p) - \frac{cst}{1-F}$ alors:
\begin{equation*}
   u(p) =
   \begin{cases}
      N*M - \frac{1}{1-F}              & p = 0 \\
      F * (c(p-1) - \frac{1}{1-F}) + cst & p \geq 1
   \end{cases}
\end{equation*}

On simplifie:
\begin{equation*}
   u(p) =
   \begin{cases}
      N*M - \frac{cst}{1-F}  & p = 0 \\
      F * c(p-1)
   \end{cases}
\end{equation*}

Alors $U_p$ est un suite géometrique ou le terme génerale est:
\begin{equation*}
   u(p) = u(0) * F^p
\end{equation*}
\begin{equation*}
   u(p) = (N * M - \frac{cst}{1-F}) * F^p
\end{equation*}

Alors on peux déduire $c(p)$:
\begin{equation*}
   c(p) = u(p) + \frac{cst}{1-F}
\end{equation*}

\begin{equation*}
   c(p) = (N * M - \frac{cst}{1-F}) * F^p + \frac{cst}{1-F}
\end{equation*}

Alors la complexité est:
\begin{equation*}
   c(p) = \mathcal{O}(F^p)
\end{equation*}

Alors si on prend $F = 20$ - \textit{estimation empirique} -
\begin{equation}
   c(p) \approx 20^p
\end{equation}

\section{Analyse expérimentale}


\end{document}
